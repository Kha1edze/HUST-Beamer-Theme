\documentclass{beamer}
\usepackage{ctex, hyperref}
\usepackage[T1]{fontenc}

% ============= 其他常用包 =============
\usepackage{latexsym,amsmath,xcolor,multicol,booktabs,calligra}
\usepackage{graphicx,pstricks,listings,stackengine}

% ============= 基本信息 =============
\author泽小润}
\title{HUST Beamer Theme}
\subtitle{毕业设计开题报告}
\institute{华中科技大学人工智能与自动化学院}
\date{2025年10月5日}

% ============= 颜色定义(华科蓝) =============
\definecolor{hustblue}{RGB}{0,62,126} % 华中科技大学官方蓝
\definecolor{deepblue}{rgb}{0,0,0.5}
\definecolor{deepred}{rgb}{0.6,0,0}
\definecolor{deepgreen}{rgb}{0,0.5,0}
\definecolor{halfgray}{gray}{0.55}

\setbeamercolor{structure}{fg=hustblue}
\setbeamercolor{frametitle}{bg=hustblue, fg=white}
\setbeamercolor{footline}{bg=hustblue, fg=white}
\setbeamercolor{title}{fg=hustblue}
\setbeamercolor{block title}{bg=hustblue, fg=white}
\setbeamerfont{frametitle}{size=\large}

% ============= 页脚格式 =============
\setbeamertemplate{footline}{
  \leavevmode%
  \hbox{%
  \begin{beamercolorbox}[wd=.8\paperwidth,ht=2.5ex,dp=1.125ex,leftskip=2ex]{author in head/foot}%
    \usebeamercolor[fg]{footline}\insertauthor
  \end{beamercolorbox}%
  \begin{beamercolorbox}[wd=.2\paperwidth,ht=2.5ex,dp=1.125ex,rightskip=2ex plus1fil]{date in head/foot}%
    \usebeamercolor[fg]{footline}\insertframenumber{} / \inserttotalframenumber
  \end{beamercolorbox}}%
  \vskip0pt%
}

% ============= 代码块样式 =============
\lstset{
    basicstyle=\ttfamily\small,
    keywordstyle=\bfseries\color{deepblue},
    emphstyle=\ttfamily\color{deepred},
    stringstyle=\color{deepgreen},
    numbers=left,
    numberstyle=\small\color{halfgray},
    rulesepcolor=\color{red!20!green!20!blue!20},
    frame=shadowbox,
}

% ============= 正文开始 =============
\begin{document}
\kaishu

% 封面页
\begin{frame}
    \titlepage
    \vspace{1em}
    \begin{figure}[htpb]
        \centering
        \includegraphics[width=0.25\linewidth]{pic/HUST_Logo.png} % 使用 .png 格式的校徽
    \end{figure}
\end{frame}

% 目录页
\begin{frame}{目录}
    \tableofcontents
\end{frame}

% ================= 内容 =================

\section{课题背景}

\begin{frame}{用Beamer做华科风PPT?}
    \begin{itemize}[<+-| alert@+>]
        \item \LaTeX{} 可以轻松做出高质量幻灯片
        \item 本模板为华中科技大学定制主题
        \item 推荐使用 Xe\LaTeX{} 编译以支持中文
        \item 模板基于 Tsinghua Beamer Theme 改编
    \end{itemize}
\end{frame}

\section{研究现状}

\begin{frame}{Beamer主题分类}
    \begin{itemize}
        \item 常见主题如 default, Madrid, Berkeley 等
        \item 各高校也有自定义主题(THU, SJTU, PKU, HUST...)
        \item 本模板旨在打造清爽、现代的华科配色风格
    \end{itemize}
\end{frame}

\section{研究内容}

\begin{frame}{HUST Theme 特点}
    \begin{itemize}
        \item 主色为 \textbf{HUST Blue}:RGB(0,62,126)
        \item 采用简洁页脚显示页码与作者信息
        \item 中文字体采用楷体,突出学术气质
        \item 支持插图、公式、代码等常用学术元素
    \end{itemize}
\end{frame}

\begin{frame}{Why Beamer?}
    \begin{itemize}
        \item \LaTeX 广泛用于学术界,论文与展示风格统一
        \item 相比 Word,更易控制公式、图表、编号
        \item PPT 与论文排版风格一致,专业且高效
    \end{itemize}
    \begin{table}[h]
        \centering
        \begin{tabular}{c|c}
            Word & \LaTeX \\
            \hline
            格式易乱 & 结构清晰 \\
            调整繁琐 & 一致性强 \\
            所见即所得 & 所见即所想 \\
        \end{tabular}
    \end{table}
\end{frame}

\section{计划进度}
\begin{frame}{项目时间表}
    \begin{itemize}
        \item 一月:文献调研与主题设计
        \item 二月:模板实现与测试
        \item 三月:优化细节与撰写报告
        \item 四月:完成答辩演示文稿
    \end{itemize}
\end{frame}

\section{参考文献}
\begin{frame}[allowframebreaks]
    \bibliography{ref}
    \bibliographystyle{alpha}
\end{frame}

\begin{frame}
    \begin{center}
        {\Huge\calligra Thank You!}\\[1em]
        \large 感谢您的聆听
    \end{center}
\end{frame}

\end{document}
